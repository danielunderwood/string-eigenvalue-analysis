\documentclass[10pt,a4paper]{article}
\usepackage[utf8]{inputenc}
\usepackage{amsmath}
\usepackage{amsfonts}
\usepackage{amssymb}
\usepackage{multicol}
\usepackage{fullpage}
\usepackage{graphicx}
\usepackage{epstopdf}
\usepackage{tikz}
\usepackage{float}
\usepackage{booktabs}
\usepackage{listings}
\usepackage{xcolor}
\usepackage[hidelinks]{hyperref}
\usepackage{cleveref}
\usepackage{bm}

\definecolor{mygreen}{RGB}{28,172,0} % color values Red, Green, Blue
\definecolor{mylilas}{RGB}{170,55,241}

\newcommand{\crefrangeconjunction}{--}
\let\oldhat\hat
\renewcommand{\vec}[1]{\boldsymbol{#1}}
\renewcommand{\hat}[1]{\oldhat{\boldsymbol{#1}}}

\author{Daniel Underwood}
\title{Eigenvalue Analysis}
\begin{document}

\lstset{language=Matlab,%
    %basicstyle=\color{red},
    breaklines=true,%
    morekeywords={matlab2tikz},
    keywordstyle=\color{blue},%
    morekeywords=[2]{1}, keywordstyle=[2]{\color{black}},
    identifierstyle=\color{black},%
    stringstyle=\color{mylilas},
    commentstyle=\color{mygreen},%
    showstringspaces=false,%without this there will be a symbol in the places where there is a space
    numbers=left,%
    numberstyle={\small \color{gray}},% size of the numbers
    numbersep=9pt, % this defines how far the numbers are from the text
    emph=[1]{for,end,break},emphstyle=[1]\color{red}, %some words to emphasise
    %emph=[2]{word1,word2}, emphstyle=[2]{style},    
}


% Parenthesis
\newcommand{\paren}[1]{\left( #1 \right)} 

\maketitle
\begin{multicols*}{2}
\section*{Vibrating String}

Many physical system can be represented as a set of masses on a string. This type of system can easily be modeled by differential equations using Newton's Second Law, resulting in the following system of differential equations:

\begin{equation}
	m_i \frac{d^2 x_i}{dt^2} = \frac{F}{h} \paren{ x_{i-1} + x_{i+1} -2 x_i}
	\label{eqn: diffeq}
\end{equation}
where $i = 1, ..., n$ for $n$ masses, $F$ is the constant horizontal tension on the string, $h$ is the horizontal separation of each of the masses, and $x_i$ is the vertical displacement for the $i^{\rm{th}}$ mass. In addition to the given values of $i$, \cref{eqn: diffeq} also indicates the displacements $x_0$ and $x_{n+1}$. These terms would represent the masses attached on the outsides of masses $i = 1$ and $i = n+1$. These cannot have a displacement, as they are not masses in the system; therefore $x_0 = x_{n+1} = 0$.

This system of differential equations can be represented by the matrix differential equation

\begin{equation}
	\frac{d^2 \vec{x}}{dt^2} = -DA \vec{x}
	\label{eqn: matrixdiffeq}
\end{equation}
where $\vec{x}$ is the column vector consisting of the vertical diaplacements, $ \left[ x_1, ..., x_n \right]^T$, and $A$ and $D$ are $n \times n$ matrices defined as

$$
A :=
	\begin{bmatrix}
		2 & -1 &  & & &  \\
		-1 & 2 & -1 &   & &  \\
		   & -1 & 2 & -1 \\
		   & & \ddots & \ddots & \ddots  \\
		   &  & & -1 & 2
	\end{bmatrix}
$$



$$
D := \frac{F}{h} 
	\begin{bmatrix}
		m_1^{-1} \\
		& m_2^{-1} \\
		& & \ddots \\
		& & & m_n^{-1}
	\end{bmatrix}
$$

\Cref{eqn: matrixdiffeq} can easily be solved by assuming a solution of the form

\begin{equation}
	\vec{x} (t) = e^{\lambda t} \vec{v}
	\label{eqn: assumedsoln}
\end{equation}

Differentiating the assumption in \cref{eqn: assumedsoln} results in

\begin{equation}
	\ddot{\vec{x}} = \lambda^2 e^{\lambda t} \vec{v} = \lambda^2 \vec{x}
	\label{eqn: differentied solution}
\end{equation}

Combining \cref{eqn: matrixdiffeq} with the derivative relation in \cref{eqn: differentied solution} results in

\begin{equation}
	\lambda^2 \vec{x} = -DA \vec{x}
	\label{eqn: eigenvalue equation 1}
\end{equation}

\Cref{eqn: eigenvalue equation 1} is the same as

\begin{equation}
	DA \vec{v} = - \lambda^2 \vec{v}
	\label{eqn: eigenvalue equation 2}
\end{equation}

by noticing that the $e^{\lambda t}$ terms cancel and the negative can be moved to $\lambda$ in order to avoid possibly iterating through a matrix and multiplying every element by $-1$.

Solving the eigenvalue in \cref{eqn: eigenvalue equation 2} results in eigenvalues $\mu = -\lambda^2$ of $DA$. Relating this with \crefrange{eqn: differentied solution}{eqn: eigenvalue equation 1} results in fundamental frequencies $\lambda = i \sqrt{\mu}$ with fundamental modes $\vec{v}$ being the eigenvectors of $DA$.

\section*{Superposition Principle with Matrix Differential Equations}

The general solution for a linear differential system can be given by

\begin{equation}
	\vec{x}\paren{t} = \sum\limits_{i=1}^n \gamma_i e^{\lambda_i t} \vec{v}_i
	\label{eqn: Superposition}
\end{equation}

\Cref{eqn: Superposition} can be expanded in terms of sines and cosines using the Euler formula, resulting in

\begin{equation}
	\vec{x}\paren{t} = \sum\limits_{i=1}^n \paren{\alpha_i \cos{\paren{\sqrt{\mu_i} t}} + \beta_i \sin{\paren{\sqrt{\mu_i} t}}} \vec{v}_i
\end{equation}
where $\lambda = -i\sqrt{\mu}$ and the coefficients $\alpha_i$ and $\beta_i$ are determinted by initial conditions

\begin{subequations}
	\begin{align}
		\vec{x} \paren{0} = \vec{x}_0 \label{eqn: xic} \\
		\dot{\vec{x}} \paren{0} = \vec{z}_0 \label{eqn: derivative ic}
	\end{align}
\end{subequations}

Luckily, cosine and sine are easily evaluated at 0, resulting in the following relations for \crefrange{eqn: xic}{eqn: derivative ic}

\begin{subequations}
	\begin{align}
		\vec{x}\paren{0} = \sum\limits_{i=1}^n \alpha_i \vec{v}_i \label{eqn: alphaeq} \\
		\dot{\vec{x}}\paren{0} = \sum\limits_{i=1}^n \beta_i \sqrt{\mu_i} \vec{v}_i \label{eqn: betaeq}
	\end{align}
\end{subequations}

By taking the eigenvectors $\{ \vec{v}_i\}$ to be column vectors in a matrix V, we have

$$
V := 
\begin{bmatrix}
	v_1^1 & v_2^1 & \cdots & v_n^1 \\
	\vdots & \vdots & & \vdots \\
	v_1^m & v_2^m & \cdots & v_n^m
\end{bmatrix}
$$

i.e., $V_i^j = v_i^j$ where $v_i^j$ is the $j^{\rm{th}}$ component of the $i^{\rm{th}}$ eigenvector or $V := [ \vec{v}_1 ... \vec{v}_n ]$. Taking $\vec{\alpha} = \left[ \alpha_1 \,\,...\,\, \alpha_n \right]^T$ and $\vec{b} = \left[ \beta_1 \sqrt{\mu_1} \,\,...\,\, \beta_n \sqrt{\mu_n} \right]^T$, we are able to rewrite \crefrange{eqn: alphaeq}{eqn: betaeq} as the following matrix equations

\begin{subequations}
	\begin{align}
		V \vec{\alpha} = \vec{x}_0 \label{eqn: amat} \\
		V \vec{b} = \vec{z}_0 \label{eqn: bmat}
	\end{align}
\end{subequations}

Using matrix techniques, the solution vectors $\alpha$ and $\vec{b}$ may be obtained. Using the definitions of these solution vectors, we hhave the coefficients $\alpha_i$ as the $i^{\rm{th}}$ component of $\vec{\alpha}$ and $\beta_i = \frac{b_i}{\sqrt{\mu_i}}$.
\end{multicols*}
\end{document}
