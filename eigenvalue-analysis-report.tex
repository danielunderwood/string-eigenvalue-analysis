\documentclass[10pt,a4paper]{article}
\usepackage[utf8]{inputenc}
\usepackage{amsmath}
\usepackage{amsfonts}
\usepackage{amssymb}
\usepackage{multicol}
\usepackage{fullpage}
\usepackage{graphicx}
\usepackage{epstopdf}
\usepackage{tikz}
\usepackage{float}
\usepackage{booktabs}
\usepackage{listings}
\usepackage{xcolor}
\usepackage[hidelinks]{hyperref}
\usepackage{cleveref}


\definecolor{mygreen}{RGB}{28,172,0} % color values Red, Green, Blue
\definecolor{mylilas}{RGB}{170,55,241}

\author{Daniel Underwood}
\title{Eigenvalue Analysis}
\begin{document}

\lstset{language=Matlab,%
    %basicstyle=\color{red},
    breaklines=true,%
    morekeywords={matlab2tikz},
    keywordstyle=\color{blue},%
    morekeywords=[2]{1}, keywordstyle=[2]{\color{black}},
    identifierstyle=\color{black},%
    stringstyle=\color{mylilas},
    commentstyle=\color{mygreen},%
    showstringspaces=false,%without this there will be a symbol in the places where there is a space
    numbers=left,%
    numberstyle={\small \color{gray}},% size of the numbers
    numbersep=9pt, % this defines how far the numbers are from the text
    emph=[1]{for,end,break},emphstyle=[1]\color{red}, %some words to emphasise
    %emph=[2]{word1,word2}, emphstyle=[2]{style},    
}

% Parenthesis
\newcommand{\paren}[1]{\left( #1 \right)} 

\maketitle
\begin{multicols*}{2}
\section*{Vibrating String}

Many physical system can be represented as a set of masses on a string. This type of system can easily be modeled by differential equations using Newton's Second Law, resulting in the following system of differential equations:

\begin{equation}
	m_i \frac{d^2 x_i}{dt^2} = \frac{F}{h} \paren{ x_{i-1} + x_{i+1} -2 x_i}
	\label{eqn: diffeq}
\end{equation}
where $i = 1, ..., n$ for $n$ masses, $F$ is the constant horizontal tension on the string, $h$ is the horizontal separation of each of the masses, and $x_i$ is the vertical displacement for the $i^{\rm{th}}$ mass. In addition to the given values of $i$, \cref{eqn: diffeq} also indicates the displacements $x_0$ and $x_{n+1}$. These terms would represent the masses attached on the outsides of masses $i = 1$ and $i = n+1$. These cannot have a displacement, as they are not masses in the system; therefore $x_0 = x_{n+1} = 0$.

This system of differential equations can be represented by the matrix differential equation

\begin{equation}
	\frac{d^2 \vec{x}}{dt^2} = -DA \vec{x}
	\label{eqn: matrixdiffeq}
\end{equation}
where $\vec{x}$ is the column vector consisting of the vertical diaplacements, $ \left[ x_1, ..., x_n \right]^T$, and $A$ and $D$ are $n \times n$ matrices defined as

$$
A :=
	\begin{bmatrix}
		2 & -1 &  & & &  \\
		-1 & 2 & -1 &   & &  \\
		   & -1 & 2 & -1 \\
		   & & \ddots & \ddots & \ddots  \\
		   &  & & -1 & 2
	\end{bmatrix}
$$



$$
D := \frac{F}{h} 
	\begin{bmatrix}
		m_1^{-1} \\
		& m_2^{-1} \\
		& & \ddots \\
		& & & m_n^{-1}
	\end{bmatrix}
$$

\Cref{eqn: matrixdiffeq} can easily be solved by assuming a solution of the form

\begin{equation}
	\vec{x} (t) = e^{\lambda t} \vec{v}
	\label{eqn: assumedsoln}
\end{equation}

Differentiating the assumption in \cref{eqn: assumedsoln} results in

\begin{equation}
	\ddot{\vec{x}} = \lambda^2 e^{\lambda t} \vec{v} = \lambda^2 \vec{x}
	\label{eqn: differentied solution}
\end{equation}

Combining \cref{eqn: matrixdiffeq} with the derivative relation in \cref{eqn: differentied solution} results in

\begin{equation}
	\lambda^2 \vec{x} = -DA \vec{x}
	\label{eqn: eigenvalue equation 1}
\end{equation}

\Cref{eqn: eigenvalue equation 1} is the same as

\begin{equation}
	DA \vec{v} = - \lambda^2 \vec{v}
	\label{eqn: eigenvalue equation 2}
\end{equation}

by noticing that the $e^{\lambda t}$ terms cancel and the negative can be moved to $\lambda$ in order to avoid possibly iterating through a matrix and multiplying every element by $-1$.

Solving the eigenvalue in \cref{eqn: eigenvalue equation 2} results in eigenvalues $\mu = -\lambda^2$ of $DA$. Relating this with \crefrange{eqn: differentied solution}{eqn: eigenvalue equation 1} results in fundamental frequencies $\lambda = i \sqrt{\mu}$ with fundamental modes $\vec{v}$ being the eigenvectors of $DA$.

\section*{Superposition Principle with Matrix Differential Equations}
\end{multicols*}
\end{document}
